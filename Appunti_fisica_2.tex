\documentclass[a4paper,12pt,titlepage,openany]{book}
\usepackage[utf8]{inputenc}
\usepackage[italian]{babel}
\usepackage[]{frontespizio}
\usepackage{amsmath}
\usepackage{amsthm}
\usepackage{amssymb}
\usepackage{mathptmx}  % Font
\usepackage{relsize}   % Displaystyle for math symbols
\usepackage{dashbox}   % Dashed box for math equations
\usepackage[linktocpage=true,hyperfootnotes=false,colorlinks=true,linkcolor=orange]{hyperref}  % Link nell'indice
\usepackage{microtype} % migliora l'uso di spazi e rientri
\usepackage{bbding}    % New symbols for itemize
\usepackage{setspace}  % Permette di personalizzare l'interlinea
\usepackage{fancyhdr}  % Gestione headings
\usepackage{graphicx}  % Permette di includere figure
\usepackage{tikz}      % Permette di disegnare
\usepackage{wrapfig}
\usetikzlibrary{arrows}
\usepackage{booktabs}  % Creazione tabelle
\usepackage{caption}   % Permette la personalizzazione delle didascalie e la loro aggiunta in tabelle
\usepackage[wide]{sidecap}  % Permette di mettere didascalie laterali a figure e tabelle
\usepackage{subfig}    % Crea sottofigure o sottotabelle
\usepackage[]{xcolor}  % Add colors

% Theorems setup
\newtheoremstyle{mydef}
{\topsep}{\topsep}%
{\leftskip=2em\rightskip=2em}{}%
{\scshape}{}%
{\newline}%
{\textbf{\thmname{#1}~\thmnumber{#2}}\thmnote{\ -\ #3}.}

\theoremstyle{mydef}
\newtheorem{definizione}{Definizione}[chapter]

% Pagestyle
\pagestyle{fancy}
\renewcommand{\headrulewidth}{0pt} % no rule at the top
\fancyhf{} % clear all fields
\fancyhead[LO,RE]{\small\bfseries\nouppercase{\leftmark}}
\fancyhead[RO,LE]{\small\bfseries\thepage}

% Other little adjustments
\renewcommand\thefootnote{\textcolor{black}{\arabic{footnote}}}
\renewcommand\labelitemi{\footnotesize\NibSolidRight}
\newcommand\dboxed[1]{\mathlarger{\dbox{\ensuremath{#1}}}}  % Allow dbox to be used in math env
\linespread{1.2}
\captionsetup{format=hang, labelfont={sf,bf}}
\sidecaptionvpos{figure}{c}  % Side caption centered to image
\newcommand{\parallels}{\mathbin{\!/\mkern-5mu/\!}} % Parallel symbol

\begin{document}
	\begin{frontespizio}
        \Preambolo{\usepackage{romannum}}
		\Universita{Verona}
        \Dipartimento{Informatica}
		\Corso[Laurea Triennale]{Informatica}
		\Annoaccademico{2017--2018}
		\Titolo{Appunti di fisica II}
		\Sottotitolo{Riassunto del corso di elettromagnetismo\\ 
                     e raccolta di formule utili per l'esame}
		\NCandidato{A cura di}
        \Candidato{Michele Martini}
        \Relatore{Prof. Claudia Daffara}
        \NRelatore{Corso tenuto da}{}
	\end{frontespizio}
    
    \frontmatter
    \pagestyle{plain}
    \cleardoublepage
    \tableofcontents
    \listoffigures
    \listoftables
    \cleardoublepage

    \chapter*{Introduzione} \addcontentsline{toc}{chapter}{Introduzione}
    \chaptermark{INTRODUZIONE}
        Il presente scritto non vuol essere una formale dispensa per il corso di fisica II,
        bensì una semplice raccolta di appunti presi a lezione, sistemati e migliorati nella
        forma e nel contenuto.
        
        Verranno dunque presentati gli argomenti nello stesso ordine con il quale sono stati
        affrontati durante le ore in università, introducendo come prime grandezze fisiche la carica
        elettrica ed il potenziale, che forniranno una solida base per poter argomentare adeguatamente
        l'elettrostatica nel vuoto.\\
        Proseguiremo successivamente trattando l'elettrodinamica, il magnetismo ed infine, unendo le
        nozioni apprese da ambo le parti, si chiuderà il cerchio trattando quindi l'elettromagnetismo.\\
        \emph{Nota:} gli argomenti verranno studiati quasi unicamente in forma integrale, approfondendo solo
        in taluni momenti la natura locale dei fenomeni analizzati.
        
        \bigskip
        La maggior parte dei fenomeni che studieremo verrà introdotta da cenni storici ed, eventualmente,
        dall'esperimento che ne ha segnato l'effettiva scoperta. Questo metodo può rivelarsi utile per vari motivi,
        tra i quali:
        \begin{enumerate}
            \item memorizzare più facilmente i passi scientifici fondamentali su cui si basa l'odierno elettromagnetismo;
            \item osservare come e con quale scopo si progettano veri e propri esperimenti di fisica;
            \item scoprire qualche nuovo, interessante aneddoto per fare colpo sulle ragazze.
        \end{enumerate}
        
        \bigskip
        \begin{center}
            Nella speranza che questo piccolo fascicolo possa esservi di qualche utilità,
            vi~auguro~\emph{buona~lettura!}
        \end{center}


    \mainmatter
    \pagestyle{fancy}
    \chapter{Elettrostatica nel vuoto}
        Nel corso dei secoli, grazie all'impegno ed agli importanti studi di innumerevoli personaggi,
        siamo giunti a racchiudere le nostre attuali conoscenze fisiche dell'universo in un unico
        modello, denominato \emph{``modello standard''}. Secondo la teoria da esso descritta,
        esistono 4 interazioni
        fondamentali, ovvero:
        \begin{itemize}
            \item forza di gravità;
            \item forza elettromagnetica;
            \item forza nucleare forte;
            \item forza nucleare debole.
        \end{itemize}
        Il MS ci dice inoltre che ognuna di esse agisce tramite un \emph{campo}; in questo capitolo studieremo
        il \emph{campo elettrico}. Tuttavia, per motivi che saranno più chiari solo nei capitoli successivi del libretto, lo studieremo in una situazione particolare, ovvero in condizioni \emph{stazionarie} (o \emph{statiche}).
        \footnote{Fenomeni non influenzati dal tempo}
        
        \section{La carica elettrica}
            Con svariati esperimenti, taluni anche molto semplici, si può facilmente osservare che ogni materiale
            può essere elettrizzato tramite strofinio, contatto o induzione elettrica. Ciò permise di supporre
            l'esistenza di una proprietà intrinseca della materia. Quest'ultima è la \emph{carica elettrica},
            e si misura in \emph{Coulomb} $[C]$.\footnote{Grandezza derivata: $1A=1C/1s$}
            
            Un'importante caratteristica della carica consiste nel fatto che sia quantizzata, infatti ha sempre valore
            multiplo di $e$.\footnote{Carica di un elettrone ($1,6022 \times 10^{-19}C$)}
            La carica di un materiale può essere inoltre positiva o negativa, e possiamo osservare quanto segue:
            \begin{itemize}
                \item cariche con segno concorde si respingono;
                \item cariche con segno discorde si attraggono.
            \end{itemize}
            \begin{figure}[htp]
                \centering
                \includegraphics[width=0.5\textwidth]{forza_coulomb}
                \captionsetup{width=0.6\textwidth, justification=raggedright}
                \caption[Attrazione/repulsione tra due cariche]{Cariche discordi di attraggono, cariche concordi si respingono.}
            \end{figure}
            Alla luce di ciò, possiamo finalmente pensare ad esperimenti pratici (con ``proprietà intrinseche'' si
            ragiona poco, ma con le forze è tutta un'altra storia!).
        
        \section{Forza di Coulomb e campo elettrostatico}
            La forza repulsiva/attrattiva che si verifica tra corpi ``carichi'' (analizzeremo in dettaglio tale
            termine in seguito) ha particolari caratteristiche, che furono studiate in modo appofondito da Charles
            Augustin de Coulomb (1736-1806) nel suo famoso esperimento eseguito con la bilancia di torsione.\\
            Sappiamo che:
            \begin{itemize}
                \item $F \propto q_1q_2$
                \item $F \propto \frac{1}{r^2}$
                \item $F$ centrale
            \end{itemize}
            Partendo da tali osservazioni, possiamo definire la forza di Coulomb.
            \begin{definizione}[Legge di Coulomb]
                Due cariche si attraggono o si respingono con una forza che agisce lungo la congiungente i centri
                dei due corpi, con intensità direttamente proporzionale alle cariche e inversamente proporzionale
                al quadrato della loro distanza.
                \begin{align}
                    \vec{F}_{1,2}=K\frac{q_1q_2}{r_{1,2}^2} \vec{u}_{1,2}\text{,}\quad
                    \text{con } k=\frac{1}{4\pi\epsilon_0}
                \end{align}
            \end{definizione}
            \noindent
            Nella definizione della forza di Coulomb incontriamo inoltre, per la prima volta, la costante $\epsilon_0$,
            \footnote{$\epsilon_0$: permittività elettrica del vuoto ($8,8541\times 10^{-12}F/m$).} la quale ci 
            accompagnerà in quasi ogni formula che analizzeremo.
            
            \emph{Vale il principio di sovrapposizione}: la forza agente tra due cariche non viene influenzata da
            eventuali altre forze presenti nel sistema ed agenti sulle medesime cariche. Con questa premessa risulta
            più che umano analizzare un sistema di $N$ cariche puntiformi.
            \begin{align*}
                \vec{F}_{tot_{q_0}} = \sum_{1}^{N}{\vec{F}_{i_0}} =
                &\underbrace{\frac{1}{4\pi\epsilon_0}\sum_{1}^{N}{\frac{q_iq_0}{r_{i,0}^2}\vec{u}_{i,0}}} =
                \frac{q_1q_0}{4\pi\epsilon_0r_{1,0}^2}\vec{u}_{1,0} + \dotsb\\
                &\vec{F}_{tot_{q_0}} = 
                \frac{\mathbf{q_0}}{4\pi\epsilon_0}\sum_{1}^{N}{\frac{q_i}{r_{i,0}^2}\vec{u}_{i,0}}\\
                &\vec{F}_{tot_{q_0}} = 
                q_0\cdot\frac{1}{4\pi\epsilon_0}\sum_{1}^{N}{\frac{q_i}{r_{i,0}^2}\vec{u}_{i,0}} =
                q_0\cdot\vec{E}(\vec{r})
            \end{align*}
            \begin{definizione}[Campo elettrostatico]
                Il campo elettrostatico $\vec{E}(\vec{r})$ è definito come la forza per unità
                di carica alla quale è soggetta una carica puntiforme $q_0$ se posta in posizione $\vec{r}$.
                \begin{align}
                    \boxed{\vec{E}(\vec{r}) = \frac{\vec{F}}{q_0}}
                \end{align}
                Tale campo è dunque una \emph{proprietà dello spazio}.
                La sua unità di misura è $[V/m]$ (oppure, ricavandola dalla formula: $[N/C]$).
            \end{definizione}
            
            \noindent
            Le linee di campo escono dalle cariche positive, dette \emph{``sorgenti''}, ed entrano nelle cariche
            negative, dette \emph{``pozzi''}.
            Sia linee di campo di $\vec{E}(\vec{r})$ sia le linee di forza della $\vec{F}$
            di Coulomb sono radiali, con centro nella sorgente del campo, e dunque parallele tra loro.
            \begin{figure}[htp]
                \centering
                \includegraphics[width=0.7\textwidth]{linee_campo}
                \caption{Linee di campo di $\vec{E}$.}
            \end{figure}
            
            \subsection{Energia e potenziale elettrostatico}
            La forza elettrostatica è \emph{conservativa}, quindi:
            \begin{itemize}
                \item{$\exists\,U\mid W=-\Delta U$;}
                \item{il lavoro che compie lungo un qualsiasi percorso chiuso è nullo;} 
                \item{il lavoro per spostare un corpo da un punto A ad un punto B dipende
                      solo dalle posizioni iniziale e finali, non dalla traiettoria.}
            \end{itemize}
            Posta una carica positiva $Q$ nell'origine e una carica $q_0$ in un punto A, il lavoro per spostare
            $q_0$ da A ad un punto B e l'energia elettrostatica delle due cariche si calcolano come segue:
            \begin{gather}
            \vec{F}_{el.st.}(r) = q_0\frac{Q}{4\pi\epsilon_0r^2}\vec{u}_r \notag \\[1ex]
            \begin{aligned}
            W_{AB} &= \int_{AB}\vec{F}_{el.st.}\cdot\,d\vec{s}
            = \int_{AB}\frac{q_0Q}{4\pi\epsilon_0r^2}\,\vec{u}_r\,d\vec{s}
            = \int_{r_A}^{r_B}\frac{q_0Q}{4\pi\epsilon_0r^2}\,dr \\
            &= -\frac{q_0Q}{4\pi\epsilon_0r}\biggr\rvert_{r_A}^{r_B}
            = -\frac{q_0Q}{4\pi\epsilon_0r_B} + \frac{q_0Q}{4\pi\epsilon_0r_A} = -\Delta U
            \end{aligned}
            \notag\\[1ex]
            \dboxed{U = \frac{q_0q_1}{4\pi\epsilon_0r} + \text{cost}}
            \end{gather}


            Se poniamo $U(\infty) = 0$, allora $U = \Delta U = U(\vec{r})-U(\infty)$.\\
            Perciò, ricordando la prima delle 3 definizioni di forza conservativa, possiamo definire $U$ come:
            \begin{itemize}
                \item{il lavoro del campo necessario per allontanare le cariche a distanza infinita;}
                \item{l'opposto del lavoro necessario a una forza esterna per portare una carica da distanza infinita dentro al sistema.}
            \end{itemize}
            
            \bigskip
            \textbf{Riassumendo}, da evidenze sperimentali siamo riusciti a trovare una forza (\emph{Forza di Coulomb} $F_{el.st.}$), dalla quale abbiamo astratto
            un'ulteriore grandezza, detta \emph{campo elettrostatico} $E_{el.st.}$, che non dipende dalla carica interessata dalla forza, bensì solo dalla
            sua posizione. In seguito, abbiamo fornito la definizione di \emph{energia potenziale} $U$. Analogamente, possiamo dunque astrarre la carica
            dalla formula dell'energia potenziale e definire una nuova proprietà dello spazio, il \emph{potenziale del campo elettrostatico}.
            
            
            \begin{gather*}
                \vec{F}_{el.st.} \;\longrightarrow\; U\\
                \Downarrow \hspace{4.5em} \Downarrow\\
                \vec{E}_{el.st.}(\vec{r}) \qquad V(\vec{r})
            \end{gather*}
            
            
            \begin{definizione}[Potenziale del campo elettrostatico]
                Il potenziale elettrostatico si definisce come l'energia potenziale elettrostatica per unita di carica.
                La sua unità di misura è il \emph{Volt} $[V]$ ed il suo valore è definito a meno di una costante.
                \begin{equation}
                \boxed{V\triangleq U/q}\\
                \end{equation}
            \end{definizione}
            
            \noindent Data la definizione di potenziale, si evince facilmente quanto segue:
            \begin{equation}\label{eqn:diff_potenziale}
                \Delta V = V_B - V_A = -\int_A^B \vec{E}\cdot d\vec{r} = -W
                \mbox{\footnotesize per carica di campo unitaria da A a B}
            \end{equation}
            Il potenziale elettrostatico è definito a meno di una costante. Perciò, grazie all'equazione
            \ref{eqn:diff_potenziale}, come abbiamo già visto per l'energia potenziale, ponendo $V(\infty) = 0$
            possiamo definire $V(\vec{r})$ come il lavoro necessario ad una forza
            esterna al campo per portare una carica \emph{unitaria} (\textbf{non} "di prova") da distanza infinita dentro al sistema.
            
            \begin{equation*}
                V(r) = -\int_\infty^r \frac{q}{4\pi \epsilon_0 r^2}dr = \dboxed{\frac{q}{4\pi \epsilon_0 r}}
            \end{equation*}
            Il campo elettrostatico punta \emph{sempre} verso potenziali \emph{decrescenti}.\\

            Vediamo come calcolare il potenziale in $\vec{r}$ quando abbiamo una \emph{distribuzione discreta}
            ($\,{q_i}$, con $i=1,2,\dots,N\,$).\\[1em]
            
            
            \begin{tikzpicture}
                % Assi
                \draw [thick,->] (0,0) -- (0,6) node[anchor=north east] {z};
                \draw [thick,->] (0,0) -- (6,0) node[anchor=north east] {y};
                \draw [thick,->] (0,0) -- (-3,-3) node[anchor=north west] {x};
                % Punti
                \filldraw [red] (4.5, 4.5) circle (2.5pt) node[anchor=south west] {r};
                \filldraw [gray] (-0.5, 4.5) circle (2pt) node[anchor=east] {$q_1$};
                \filldraw [gray] (-0.5, 2) circle (2pt) node[anchor=east] {$q_2$};
                \filldraw [gray] (1, 3.5) circle (2pt) node[anchor=east] {$q_3$};
                \filldraw [gray] (2, -1) circle (2pt) node[anchor=east] {$q_4$};
                \filldraw [gray] (4, 2.5) circle (2pt) node[anchor=north west] {$q_5$};
                % Vettori congiungenti
                \draw [gray,-stealth] (-0.5, 4.5) -- (4.5,4.5) node[midway,above,sloped] {$\vec{r_1}$};
                \draw [gray,-stealth] (-0.5, 2) -- (4.5,4.5);
                \draw [gray,-stealth] (1, 3.5) -- (4.5,4.5);
                \draw [gray,-stealth] (2, -1) -- (4.5,4.5);
                \draw [gray,-stealth] (4, 2.5) -- (4.5,4.5);
            \end{tikzpicture}
            
            \noindent
            Dato che $\vec{E} = \sum \vec{E}_i$, possiamo facilmente notare che $V=\sum V_i$.\\
            Perciò, con un sistema di $N$ cariche puntiformi, otteniamo:
            \begin{equation}
                V(r) = \frac{1}{4\pi\epsilon_0} \,\sum\frac{q_i}{r_i}
            \end{equation}
            
            Se passiamo invece da una distribuzione discreta ad una \emph{distribuzione continua},
            dobbiamo distinguere 3 tipi di distribuzioni.
            \begin{gather*}
                \text{Prima di tutto, ricordando che } \sum \rightarrow \int \text{, possiamo dire:}\\[1em]
                V(r)=\frac{q}{4\pi\epsilon_0}\,\int\frac{\mathop{dq}}{r}\\
                q_i\longmapsto \mathop{dq} \longrightarrow
                \lambda\mathop{dl}\,|\,\sigma\mathop{dS}\,|\,\rho\mathop{dV}\\[1em]
                \text{Di conseguenza il potenziale si calcolerà come segue:}\\
                \lambda\mathop{dl} \Rightarrow
                V(r)=\frac{q}{4\pi\epsilon_0}\,\int\frac{\lambda}{r}\mathop{dl}
                \quad\text{\small (distribuzione lineare)}\\
                \sigma\mathop{dS} \Rightarrow
                V(r)=\frac{q}{4\pi\epsilon_0}\,\int\frac{\sigma}{r}\mathop{dS}
                \quad\text{\small (distribuzione superficiale)}\\
                \rho\mathop{dV} \Rightarrow
                V(r)=\frac{q}{4\pi\epsilon_0}\,\int\frac{\rho}{r}\mathop{dV}
                \quad\text{\small (distribuzione volumetrica)}\\
            \end{gather*}
            
            \subsection{Teorema di Gauss}
            Prima di poter affrontare questo importante teorema, che useremo poi nel 90\% degli
            esercizi, ci occorrono alcune nozioni di base:
            \begin{itemize}
                \item{flusso attraverso una superficie;}
                \item{angolo solido.}
            \end{itemize}
            Il \emph{flusso elementare} del campo attraverso $\mathop{dS}$ è definito come segue:
            \begin{equation*}
                \mathop{d\phi} = \vec{E}\cdot\mathop{d\vec{S}}
            \end{equation*}
            \begin{center}
                (ricordando che $\mathop{d\vec{S}} = \mathop{dS}\cdot\hat{n}$)
            \end{center}
            
            Il valore del flusso dipende dalla direzione del campo elettrico e della superficie,
            ovvero della sua normale, ed in particolare dall'angolo che si forma tra essi.\\
            Il flusso avrà perciò il seguente valore:
            \begin{itemize}
                \item{
                    Superficie perpendicolare al campo ($\vec{E}\parallels\hat{n}$):\\
                    \begin{center}
                        \begin{tikzpicture}
                        \draw [-stealth] (0,0.5) -- (1,0.5) node[midway, above, sloped] {$\vec{E}$};
                        \draw [-stealth] (0,0.3) -- (1,0.3);
                        \draw [-stealth] (0,0.1) -- (1,0.1);
                        \draw (1.2, 0.7) -- (1.2, -0.2);
                        \draw (1.2, 0.7) -- (1.5, 0.8);
                        \draw (1.2, -0.2) -- (1.5, -0.1);
                        \draw (1.5, 0.8) -- (1.5, -0.1);
                        \draw [-stealth] (1.35,0.3) -- (2, 0.3) node[anchor=south west] {$\hat{n}$};
                        \end{tikzpicture}
                    \end{center}
                    \begin{equation*}
                        \mathop{d\phi} = \vec{E}\cdot\mathop{d\vec{S}} = E\cdot\mathop{dS}
                    \end{equation*}
                }
                \item{
                    Superficie parallela al campo ($\vec{E}\perp\hat{n}$):\\
                    \begin{center}
                        \begin{tikzpicture}
                        \draw [-stealth] (0,0.5) -- (1,0.5) node[midway, above, sloped] {$\vec{E}$};
                        \draw [-stealth] (0,0.3) -- (1,0.3);
                        \draw [-stealth] (0,0.1) -- (1,0.1);
                        \draw (1.2, 0.2) -- (1.4, 0.5);
                        \draw (1.2, 0.2) -- (2, 0.2);
                        \draw (1.4, 0.5) -- (2.2, 0.5);
                        \draw (2, 0.2) -- (2.2, 0.5);
                        \draw [-stealth] (1.7,0.35) -- (1.7, 0.7) node[anchor=south west] {$\hat{n}$};
                        \end{tikzpicture}
                    \end{center}
                    \begin{equation*}
                    \mathop{d\phi} = \vec{E}\cdot\mathop{d\vec{S}} = 0
                    \end{equation*}
                }
                \item{
                    Superficie forma un angolo $\theta\neq k\frac{\pi}{2}$ con il campo:\\
                    \begin{center}
                        \begin{tikzpicture}
                        \draw [-stealth] (0,0.5) -- (1,0.5) node[midway, above, sloped] {$\vec{E}$};
                        \draw [-stealth] (0,0.3) -- (1,0.3);
                        \draw [-stealth] (0,0.1) -- (1,0.1);
                        \draw (1.2, 0.7) -- (1.0, -0.2);
                        \draw (1.2, 0.7) -- (1.6, 0.7);
                        \draw (1, -0.2) -- (1.4, -0.2);
                        \draw (1.6, 0.7) -- (1.4, -0.2);
                        \draw [dashed] (1.3,0.3) -- (2.2, 0.3);
                        \draw [-stealth] (1.3,0.3) -- (2.2, -0.1) node[anchor=north east] {$\hat{n}$};
                        \draw (1.9,0.05) arc (-15:0:1) node[midway, anchor=west] {\scriptsize $\theta$};
                        \end{tikzpicture}
                    \end{center}
                    \begin{equation*}
                    \mathop{d\phi} = \vec{E}\cdot\mathop{d\vec{S}} = E\cdot\mathop{dS}\cdot\cos\theta
                    \end{equation*}
                }
            \end{itemize}
        
            Da queste osservazioni possiamo finalmente arrivare ad una definizione più precisa del flusso.
            \begin{definizione}[Flusso del campo elettrico]
                Il flusso del campo elettrico attraverso una superficie S è dato da
                \begin{equation}
                    \phi(\vec{E}) = \int\limits_{S} \vec{E}\cdot\mathop{d\vec{S}} \qquad [\phi] = V\cdot m
                \end{equation}
            \end{definizione}
            
            \noindent
            Prendiamo ora in considerazione una superficie chiusa qualsiasi $S$ al cui interno poniamo una
            carica puntiforme $q$. Guardando una porzione infinitesima di $S$, calcoliamo che
            il flusso attraverso essa è il seguente:
            \begin{equation}\label{eqn:angolo_solido}
                \mathop{d\phi} = \vec{E}\cdot\mathop{d\vec{S}} =
                \frac{q}{4\pi\epsilon_0r^2}\cdot\vec{u}_r\cdot\hat{n}\cdot\mathop{dS} \Rightarrow
                \frac{q}{4\pi\epsilon_0}\cdot
                \dboxed{\frac{\vec{u}_r\cdot\hat{n}\cdot\mathop{dS}}{r^2}}
            \end{equation}
            La parte di formula evidenzata nella \ref{eqn:angolo_solido} indica l'\emph{angolo solido}
            $\mathop{d\Omega}$, ed è importante in quanto il flusso dipende dall'angolo solido,
            \emph{non} dalla superficie.
            
            \begin{equation}\label{eqn:flusso_sup_chiusa}
                \phi = \int\limits_S \mathop{d\phi} = \frac{q}{4\pi\epsilon_0}\int\limits_S\mathop{d\Omega} = \frac{q}{4\pi\epsilon_0}\cdot 4\pi = \frac{q}{\epsilon_0}
                \,\Rightarrow\,\dboxed{\phi=\frac{q}{\epsilon_0}}
            \end{equation}
            
            Con il risultato del'equazione \ref{eqn:flusso_sup_chiusa}, che ci indica mostra il flusso del campo $\vec{E}$, generato da una carica puntiforme $q$, attraverso
            una superficie chiusa, possiamo studiare il caso di una distribuzione discreta di cariche puntiformi come segue.
            \begin{gather*}
                \text{Ricordando che con distribuzioni discrete abbiamo }\\
                \vec{E} = \sum\vec{E}_i\\
                \text{Calcoliamo il flusso }\phi\\
                \phi = \int\limits_S\mathop{d\phi} = \int\limits_S\sum\vec{E}_i\mathop{dS} \,\Rightarrow\, \dboxed{\phi=\frac{\sum q_i}{\epsilon_0}}
            \end{gather*}
            
            \noindent
            Abbiamo finalmente tutti gli strumenti per vedere e comprendere la seconda delle \emph{equazioni di Maxwell}.
            \begin{definizione}[Teorema di Gauss]
                Il flusso del campo elettrico attraverso una qualsiasi superficie chiusa è pari alla somma algebrica delle cariche interne
                diviso la costante dielettrica del vuoto.
                \begin{equation}
                    \boxed{\phi(\vec{E}) = \oint\limits_{\substack{S\\\text{chiusa}\\\text{qualunque}}}\vec{E}\cdot\mathop{d\vec{S}} = \frac{Q_{\text{interna}}}{\epsilon_0}}
                \end{equation}
            \end{definizione}
            \noindent
            Il teorema di Gauss è \emph{generale}, non è valido solo per il campo elettrostatico, bensì anche per il campo elettrico in condizioni \emph{non stazionarie}!
            
            Usiamolo per calcolare il campo $\vec{E}_{el.st.}$ di una carica $q$, giusto per vedere come funziona nella situazione più semplice. Consideriamo quindi una
            carica, $q$, ed un punto dove calcolare il campo, $\vec{r}$.\\
            \begin{tikzpicture}
                \filldraw (0,0) circle (2pt) node[anchor=south west] {q};
                \draw [-stealth] (0,0) -- (1, -0.6) node[anchor=south] {$\vec{r}$};
            \end{tikzpicture}\\
            $\vec{E}(\vec{r})$?\\
            A questo punto osserviamo una cosa molto importante: siamo in condizioni di \emph{simmetria sferica}. Dato che, per usare il teorema di Gauss, dobbiamo
            scegliere una superficie chiusa qualunque, questo semplice indizio ci consente di semplificarci la vita usando $S_{\text{Gauss}}=\,$sfera.\\[1em]
            \begin{tikzpicture}
                \filldraw (0,0) circle (2pt) node[anchor=south west] {q};
                \draw [-stealth] (0,0) -- (1, -0.6) node[anchor=south] {$\vec{r}$};
                \draw [gray, dashed] (0,0) circle (1.2);
                \draw [gray] (0,-1.2) node[anchor=north] {$S(r)$};
            \end{tikzpicture}\\
            Applichiamo quindi il teorema:
            \begin{align*}
                \phi(\vec{E}) \;= &\oint\limits_{S(\vec{r})}\vec{E}\cdot\mathop{d\vec{S}} = \oint\limits_{S(\vec{r})}E\cdot\mathop{dS}\;
                \mbox{\footnotesize ($\vec{u}_r$ e $\hat{n}$ sono paralleli $\Rightarrow\,\vec{u}_r \cdot \hat{n} = 1$)}\\
                &E(\vec{r})\cdot\oint\limits_{S(\vec{r})}\mathop{dS} = E\cdot4\pi r^2 = \;\frac{q}{\epsilon_0}\\[1em]
                &\dboxed{E(r) = \frac{q}{4\pi\epsilon_0 r^2}}\\
                \mbox{\footnotesize Direzione: }& \mbox{\footnotesize radiale.\quad Verso: $q^+$ uscente, $q^-$ entrante}
            \end{align*}
            \begin{tikzpicture}
                \draw [thick, ->] (0,0) -- (0,2.2) node[anchor=south east] {$E(r)$};
                \draw [thick, ->] (0,0) -- (3.2,0) node[anchor=north west] {$r$};
                \draw (0.1,1.8) .. controls (0.4,0.7) and (1,0.3) .. (2.8,0.2) node[anchor=south] {$\propto\frac{1}{r^2}$};
                \draw [densely dashed] (0.1,1.8) -- (0.07, 2.3);
                \draw [densely dashed] (2.8,0.2) -- (3.3, 0.17);
            \end{tikzpicture}
            
            \bigskip\noindent
            Fare riferimento all'esercizio \ref{es:1} per un esempio completo riguardo al calcolo di $E$ e $V$.
            
    \chapter{Elettrodinamica}
        In questo capitolo studieremo i fenomeni elettrici in condizioni \emph{non stazionarie}, introducendo la \emph{corrente elettrica} e studiandone il comportamento nei conduttori.
        
        \section{Conduzione elettrica}
            All'interno dei metalli, gli elettroni di valenza sono liberi di muoversi e non sono legati ad un atomo
            specifico. Energia cinetica media di tali cariche è data dall'\emph{``agitazione termica''}:
            \begin{align}
                \overline{E}_K &= 3K\frac{T}{2} \notag \\
                &= \frac{1}{2}\,m\,\overline{v}^2 \label{eqn:vel_termica}
            \end{align}
            La $v$ presente in \ref{eqn:vel_termica} è la \emph{velocità termica} delle particelle: ha direzione casuale
            e misura circa $1,2\times 10^5 m/s$.\footnote{In un conduttore con temperatura $300K$.}
            
            Se immergiamo il metallo in un campo $\vec{E}$ generiamo un moto ordinato nella nuvola di elettroni,
            detto \emph{velocità di deriva}: essa ha una direzione ben precisa ed una velocità generalmente molto
            più bassa rispetto a quella termica (una differenza di svariati ordini di grandezza!).\\
            Tale moto ordinato è chiamato \emph{``conduzione elettrica''} o \emph{``corrente''}.
            
            \begin{definizione}[Corrente]
                Moto ordinato degli elettroni di un conduttore. Considerando un conduttore di sezione $S$ percorso da
                corrente, l'intensità di quest'ultima si misura in \emph{Ampere} $[A]$ e si definisce come la quantità
                di carica $dQ$ che attraversa la superficie $S$ in un intervallo di tempo $dt$:
                \begin{equation}
                    \boxed{I = \frac{dQ}{dt}}
                \end{equation}
                Per convenzione e ragioni storiche si indica con segno positivo il verso di moto delle cariche positive.
            \end{definizione}
            
            \begin{SCfigure}[0.3][htp]
                \centering
                \includegraphics[width=0.7\textwidth]{corrente_elettrica}
                \captionsetup{format=plain, justification=raggedright,labelsep=newline}
                \caption{Segno della corrente elettrica\\ secondo la convenzione.}
            \end{SCfigure}
            
            \noindent
            Conoscendo l'intensità di corrente $I$ che scorre in un conduttore, data una superficie $S$, possiamo ricavarne
            la \emph{densità di corrente $\vec{J}$}.
            \begin{align}
                I = \int_{\sup}\vec{J}\cdot\,d\vec{S}& \notag\\
                \text{Flusso attraverso} & \text{ la superficie} \notag\\
                \overbrace{\,n\cdot e\cdot\vec{v}\cdot d\vec{S}\,} = &\frac{dq}{dt} \text{\ ,\ \ dove }
                n =\frac{\text{\# elettroni}}{\text{volume}} \notag\\[1em]
                \dboxed{\vec{J} = n\cdot e\cdot\vec{v}} & \qquad \dboxed{J = \frac{I}{\Sigma}}
            \end{align}

    \chapter{Magnetostatica}
        Evidenze osservate nei vari esperimenti:
        \begin{itemize}
            \item alcuni materiali (composti da magnetite) possono attirare altri materiali (ferrosi);
            \item alcuni materiali si orientano lungo una direzione privilegiata;
            \item i fenomeni attrattivi/repulsivi di cui sopra si manifestano sui bordi ($\vec{F}$~localizzata);
            \item \emph{non esistono cariche magnetiche isolate}. Coulomb tentò di riprodurre lo stesso esperimento
                della bilancia di torsione, ma senza alcun risultato;
            \item se spezzo un magnete, ottengo due nuovi magneti: esistono solo dipoli magnetici;
            \item \emph{esperimento di \O{}rsted} permise di scoprire che le correnti sono sorgenti del campo magnetico;
            \item ad Ampere dobbiamo la scoperta che queste sono le \emph{uniche} sorgenti del campo magnetico.
        \end{itemize}
    
    \appendix
    \chapter{Esercizi}
    \section{Elettrostatica}
        \subsection{Superficie sferica con distribuzione superficiale}\label{es:1}
            \begin{wrapfigure}[6]{r}{5cm}
                \begin{tikzpicture}
                \draw [gray] (0,0) circle (1);
                \draw [gray] (0,0) -- (0.7,0.7) node[midway, above] {R};
                \draw [gray] (0.5,-0.9) node[anchor=north] {$q=10^{-9}C$};
                \end{tikzpicture}
            \end{wrapfigure}
            Consideriamo una distribuzione superficiale di carica $q=10^{-9}C$ su una superficie sferica di raggio $R=5cm$.
            Vogliamo conoscere $E(\vec{r})$ e $V(\vec{r})$ ovunque nello spazio (dentro e fuori la sfera).\\
            
            Dato che stiamo studiando una superficie sferica, possiamo usare come $S_{\text{Gauss}}$ una sfera. Prendiamone dunque una
            di raggio $r$ ed applichiamo il teorema.\\
            
            \begin{tikzpicture}
                \draw (0,0) circle (1.5);
                \draw [dashed,gray] (0,0) circle (1.1);
                \draw [dashed,gray] (0,0) circle (1.8);
                \draw (0,0) -- (1.05,1.05) node[midway, above] {R};
                \draw [gray] (0,0) -- (0,-1.1) node[midway, left] {r};
                \draw [gray] (0,0) -- (1.6,-0.75) node[midway, above] {r};
            \end{tikzpicture}
            
            Per Gauss:
            \begin{equation*}
                \phi(\vec{E}) = \oint\limits_{S(r)}=E(r)\cdot\oint\limits_{S(r)}\mathop{dS} = E(r)\cdot\ 4\pi r^2 = \frac{Q_{\text{int}}}{\epsilon_0}
            \end{equation*}
            da cui:
            \begin{itemize}
                \item[($r<R$)]{
                    La somma delle cariche interne alla superficie di Gauss (raggio $r$) è nulla, ovvero $q=0$.
                    Da ciò ricaviamo che:
                    \begin{equation*}
                        E(r)\cdot 4\pi r^2 = \frac{0}{\epsilon_0}\,\Rightarrow\,E(r) = 0
                    \end{equation*}
                }
                \item[($r>R$)]{
                    La somma delle cariche interne alla superficie di Gauss è data dalla carica $q$ distribuita sulla
                    superficie della sfera di raggio $R$, perciò:
                    \begin{equation*}
                    E(r)\cdot 4\pi r^2 = \frac{q}{\epsilon_0}\,\Rightarrow\,E(r) = \frac{q}{4\pi\epsilon_0 r^2}
                    \end{equation*}
                }
                \item[($r=R$)]{
                    In particolare, possiamo dire che appena fuori dalla superficie della sfera di raggio $R$, il campo misurato è pari a:
                    \begin{equation*}
                    E(R)\cdot 4\pi R^2 = \frac{q}{\epsilon_0}\,\Rightarrow\,E(R) = \frac{q}{4\pi\epsilon_0 R^2} = \frac{\sigma}{\epsilon_0}
                    \;\mbox{ con }\sigma=\frac{q}{\mbox{sup.sfera}}
                    \end{equation*}
                }
            \end{itemize}
            Per precisione e completezza, calcoliamo anche la densità di carica superficiale $\sigma$.
            \begin{equation*}
                \sigma=\frac{q}{S}=\frac{q}{4\pi R^2} = \frac{10^{-9}}{4\pi 25\cdot 10^{-4}} \approx 8\cdot 10^{-6}\, C/m^2
            \end{equation*}
            
            \bigskip
            Studiamo ora il potenziale del campo. Dato che esso è definito a meno di una costante, decidiamo di considerare come costante $V(\infty) = 0$.
            \begin{gather*}
                \text{Possiamo calcolare la differenza di potenziale come segue:}\\
                \Delta V = V(r_2) - V(r_1) = -\int_{r_1}^{r_2} \vec{E}\cdot\mathop{d\vec{r}} = -\int_{r_1}^{r_2}\frac{q}{4\pi\epsilon_0 r^2}\mathop{dr}\\
                \frac{q}{4\pi\epsilon_0 r}\bigg|_{r_1}^{r_2} = \frac{q}{4\pi\epsilon_0 r_2} - \dboxed{\frac{q}{4\pi\epsilon_0 r_1}}\longrightarrow\mbox{ (cost)} \\
                \text{Il potenziale in $r > R$ in questo modo risulta:}\\
                V(r) = \frac{q}{4\pi\epsilon_0 r} + \mbox{cost}\\
                \text{Non siamo convinti? Ricalcoliamo ricordando di porre $V(\infty) = 0$:}\\
                V(r) = V_\infty -\int_{\infty}^{r}E(r)\cdot\mathop{dr} = 0 - \int_{\infty}^{r}\frac{q}{4\pi\epsilon_0 r^2}\cdot\mathop{dr}=\frac{q}{4\pi\epsilon_0 r}\\
                \text{Il potenziale in $r<R$ rimane invariato in quanto, per l'assenza di campo $E$, $\Delta V = 0$:}\\
                \Delta V = 0 \longrightarrow V = cost \longrightarrow V(r) = V(R) \quad\mbox{per } r<R
            \end{gather*}
            
            \noindent
            Mostriamo graficamente l'andamento del campo elettrostatico e del potenziale.\\
            
            \begin{tikzpicture}
                % Campo elettrostatico
                \draw [thick, ->] (0,0) -- (0,3) node[anchor=south east] {E};
                \draw [thick, ->] (0,0) -- (4,0) node[anchor=north] {r};
                \draw [dashed] (1,3) -- (1,0) node[anchor=north] {R};
                \draw [thick,red] (0,0) -- (1,0);
                \draw [thick,red] (1,2) .. controls (1.5, 0.6) and (2, 0.3) .. (3.8, 0.2) node[anchor=south] {$\propto\frac{1}{r^2}$};
                
                % Potenziale
                \draw [thick, ->] (6,0) -- (6,3) node[anchor=south east] {V};
                \draw [thick, ->] (6,0) -- (10,0) node[anchor=north] {r};
                \draw [dashed] (7,3) -- (7,0) node[anchor=north] {R};
                \draw [thick,red] (6,2) -- (7,2);
                \draw [thick,red] (7,2) .. controls (7.5, 1.1) and (8.5, 0.6) .. (9.8, 0.2) node[anchor=south] {$\propto\frac{1}{r}$};
            \end{tikzpicture}

    
    \chapter{Costanti fisiche}
        \begin{table}[h]
            \caption{Principali costanti fisiche.}
            \centering
            \begin{tabular}{lcrc}
                \toprule
                Costante fisica & Simbolo & Valore & Unità di misura\\
                \midrule
                Velocità della luce nel vuoto & $c$ & $299\,792\,458$ & $m\,s^{-1}$\\
                Costante di Plank & $h$ & $6,6260\times 10^{-34}$ & $J\,s$\\
                Carica dell'elettrone & $e$ & $1,6022\times 10^{-19}$ & $C$\\
                Massa dell'elettrone & $m_e$ & $9,1094\times 10^{-31}$ & $kg$\\
                Costante dielettrica del vuoto & $\epsilon_0$ & $8,8542\times 10^{-12}$ & $F\,m^{-1}$\\
                Permeabilità magnetica del vuoto & $\mu_0$ & $12,5664\times 10^{-7}$ & $N\,A^{-2}$\\
                Costante di Boltzman & $k$ & $1,3807\times 10^{-23}$ & $J\,K^{-1}$\\
                Numero di Avogadro & $N_A$ & $6,0221\times 10^{-23}$ & mol$^{-1}$\\
                \bottomrule
            \end{tabular}
        \end{table}
    
    \chapter{Costanti dielettriche e magnetiche}
        \begin{table}[ht]
            \captionsetup{width=0.7\textwidth}
            \caption{Costante dielettrica relativa e permeabilità magnetica relativa di alcune sostanze.}
            \centering
            \subfloat[][\emph{Costanti dielettriche relative di alcune sostanze.}]{
                \begin{tabular}{lc}
                    \toprule
                    Materiale & Costante dielettrica relativa $[\epsilon]$\\
                    \midrule
                    Vuoto & $1$\\
                    Aria secca & $1,00059$\\
                    Elio & $1,00087$\\
                    Acqua & $80$\\
                    Glicerina & $43$\\
                    Benzene & $3,1$\\
                    Carta & $3,5$\\
                    Polistirolo & $2,6$\\
                    Bachelite & $4,9$\\
                    \bottomrule
                \end{tabular}
            }
            \subfloat[][\emph{Permeabilità magnetica relativa di alcune sostanze.}]{
                \begin{tabular}{lc}
                    \toprule
                    Materiale & Permeabilità magnetica relativa $[\mu]$\\
                    \midrule
                    Vuoto & $1$\\
                    Oro & $0,999964$\\
                    Argento & $0,999974$\\
                    Rame & $0,9999902$\\
                    Acqua & $0,9999912$\\
                    Aria & $1,0000004$\\
                    Platino & $1,000360$\\
                    Ferro & $5,50$\\
                    Permalloy & $25,00$\\
                    \bottomrule
                \end{tabular}
            }
        \end{table}
\end{document}




